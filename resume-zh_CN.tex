% !TEX TS-program = xelatex
% !TEX encoding = UTF-8 Unicode
% !Mode:: "TeX:UTF-8"

\documentclass{resume}
\usepackage{zh_CN-Adobefonts_external} % Simplified Chinese Support using external fonts (./fonts/zh_CN-Adobe/)
% \usepackage{NotoSansSC_external}
% \usepackage{NotoSerifCJKsc_external}
% \usepackage{zh_CN-Adobefonts_internal} % Simplified Chinese Support using system fonts
\usepackage{linespacing_fix} % disable extra space before next section
\usepackage{cite}

\begin{document}
\pagenumbering{gobble} % suppress displaying page number

\name{佘嘉洛}

\basicInfo{
  \email{shejialuo@gmail.com} \textperiodcentered\
  \phone{(+86) 13019582712} \textperiodcentered\
  \github[shejialuo]{https://www.github.com/shejialuo}
}

\section{\faGraduationCap\  教育背景}
\datedsubsection{\textbf{西安电子科技大学},西安,陕西}{2021 -- 至今}
\textit{在读硕士研究生}\ 软件工程,预计2024年6月毕业
\datedsubsection{\textbf{重庆交通大学}, 重庆}{2017 -- 2021}
\textit{学士}\ 通信工程

\section{\faCogs\ IT 技能}
% increase linespacing [parsep=0.5ex]
\begin{itemize}[parsep=0.5ex]
  \item \textbf{编程语言:}C++ == C > Haskell > Go == Js == Python > Java == Scala
  \item \textbf{平台:}熟悉Linux系统,基础的shell脚本及python编程,能使用基本的运维工具docker,k8s,istio。
  \item \textbf{体系结构:}了解基本的分布式系统与并行计算。
  \item \textbf{函数式:}了解基本的函数式编程语言,掌握函数式编程思维。
\end{itemize}

\section{\faUsers\ 项目经历}

\datedsubsection{\textbf{嵌入式系统分布式集群管理技术}}{2021年9月 -- 2022年12月}
\role{C++,项目主负责人}{西安航空工业计算所合作项目}
\begin{onehalfspacing}
针对航空计算机,提供边车注入、服务发现及负载均衡等分布式集群管理技术。
\begin{itemize}
  \item 模仿k8s,利用准入控制器概念实现自动边车注入。
  \item 实现基于轮询和地域的负载均衡算法及基于阈值的弹性伸缩算法。
\end{itemize}
\end{onehalfspacing}

\datedsubsection{\textbf{CS144:实现TCP/IP协议栈}}{2022年6月 -- 2022年10月}
\role{C++,Linux}{个人项目}
\begin{onehalfspacing}
实现基本的TCP/IP协议栈,https://github.com/shejialuo/CS144
\begin{itemize}
  \item 基于RingBuffer的思想实现字节流的重排。
  \item 实现了除拥塞控制外的基础功能的TCP/IP协议。
  \item 通过英文实现了撰写思路:https://luolibrary.com/categories/CS144/
\end{itemize}
\end{onehalfspacing}

\datedsubsection{\textbf{ugit-cpp: 实现一个简易的git}}{2022年4月 -- 2022年10月}
\role{C++,Linux}{个人项目}
\begin{onehalfspacing}
简易的git实现,https://github.com/shejialuo/ugit-cpp
\begin{itemize}
  \item 模仿git基于Content-hash存储blob,tree及commit。
  \item 实现了分支,标签等除远程外的基本功能。
\end{itemize}
\end{onehalfspacing}

\datedsubsection{\textbf{CS149: 并行计算}}{2022年5月 -- 2022年12月}
\role{C++,CUDA,openMP}{个人项目}
\begin{onehalfspacing}
并行计算,https://github.com/shejialuo/CS149-fall21
\begin{itemize}
  \item 使用C++实现一个能解析任务依赖关系的任务执行库。
  \item 使用CUDA进行并行渲染。
  \item 使用openMP实现大图的广度优先搜索算法。
  \item 通过英文实现了撰写思路:https://luolibrary.com/categories/CS149/
\end{itemize}
\end{onehalfspacing}

% Reference Test
%\datedsubsection{\textbf{Paper Title\cite{zaharia2012resilient}}}{May. 2015}
%An xxx optimized for xxx\cite{verma2015large}
%\begin{itemize}
%  \item main contribution
%\end{itemize}

\section{\faBook\ 技术文章}

\datedsubsection{\textbf{Database Systems Complete book solutions}}{2021年12月 -- 至今}
\begin{onehalfspacing}
https://github.com/shejialuo/database-systems-complete-book-solutions
\end{onehalfspacing}

\datedsubsection{\textbf{Kubernetes client-go 源码解析——WorkQueue}}{}
\begin{onehalfspacing}
https://luolibrary.com/2022/11/14/Kubernetes-client-go-源码解析——WorkQueue
\end{onehalfspacing}

\datedsubsection{\textbf{ZStack OSAL分析}}{}
\begin{onehalfspacing}
https://luolibrary.com/2021/03/06/ZStack-OSAL分析
\end{onehalfspacing}

\section{\faHeartO\ 获奖情况}
\datedline{四川省普通高中优秀学生干部}{2017年3月}
\datedline{\textit{本科组二等奖}, 高教社杯全国大学生数学建模竞赛}{2019年11月}
\datedline{本科生国家奖学金}{2020年12月}

\section{\faInfo\ 其他}
% increase linespacing [parsep=0.5ex]
\begin{itemize}[parsep=0.5ex]
  \item 技术博客: https://luolibrary.com
  \item GitHub: https://github.com/shejialuo
  \item 语言: 英语 - 熟练(6级-551分)
\end{itemize}

%% Reference
%\newpage
%\bibliographystyle{IEEETran}
%\bibliography{mycite}
\end{document}
